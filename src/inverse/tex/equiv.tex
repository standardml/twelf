
\subsubsection{A Note on Implementing Equivalence Checking}

  For various reasons, we need equivalence checking to
be as fast as possible.  Equivalence checking is complicated
by notational definitions.  If checking $A=B$ fails, we might need
to expand definitions in one or both terms.  One could simply
expand all definitions to yield a sound algorithm, but this would
be horribly slow.  

  The solution given by Twelf, suggested by 
Pfenning and Reed, is to store two extra bits of information 
with each constant.  The first is the \emph{height} of a constant.
This merely records the definition depth of a constant.  Constants that
do not refer to other constants have height 0.  A constant $c$ that refers
to others has height $1 + \max\set{\mbox{height of constants occuring in } c}$.
Note that only constants with the same height can be equal.  
The second bit of data is the \emph{root}, it{i.e.} the head of the term that would be obtained by 
expanding all definitions.  Note that any two equal terms must have the
same root after full expansion. 

These two bits of data yield a natural algorithm for determining 
equivalence of terms.  If, when checking equality of $A=B$,
we find that two constants must be equal for $A,B$ to
be equivalent, but are not syntactically equal, 
we check to see if the roots are the same.
If not, we fail.  Otherwise, we check the heights.  If the heights
differ, we expand the constant with the greater height until 
they are equal, and check again.  Once the levels
are equal, while the constants are still distinct, we 
expand both definitions, level by level, until all constants
are expanded.  In the worst case, this will take as much time
as expanding all the definitions.  In the usual case, however,
where the constants differ\footnote{indeed, unification fails
around \%80 of the time}, the clash will be found long before
the terms are fully expanded.  

This leads to an additional rule 

\newcommand{\Root}{\mbox{root}}
$$
\begin{array}{cccc}
\infer{\Equiv{c\cdot S}{c'\cdot S'}}{\Root(c)=\Root(c') & \card{c} \geq \card{c'} & c\StepsTo M' & \Equiv{M'@S}{c'\cdot S'}} 
\end{array} 
$$

where $\card{c}$ is the height of $c$, and $\Root(c)$ is the root.
